\documentclass[
    parskip=half,
    toc=flat,
    toc=sectionentrydotfill,
]{scrartcl}  % KOMA-Script :o

\usepackage{amsmath}
\usepackage{amssymb}
\usepackage{amsthm}

\usepackage{enumitem}
\setlist[enumerate,1]{label={(\roman*)}}

\usepackage{fontspec}
\usepackage{unicode-math}
\setmainfont{TeX Gyre Termes}
\setmathfont{STIX Two Math}

\usepackage{geometry}
\geometry{letterpaper, margin=1in}

\usepackage{hyperref}

\linespread{1.236}

\title{Galois Theory}
\subtitle{MATH 440}
\author{Steven Xia}

\theoremstyle{definition}
\newtheorem{definition}{Definition}[section]
\newtheorem{subdefinition}{Definition}[definition]

\theoremstyle{plain}
\newtheorem{theorem}{Theorem}[section]
\newtheorem{corollary}{Corollary}[theorem]
\newtheorem{lemma}{Lemma}[section]

\theoremstyle{remark}
\newtheorem{remark}{Remark}[section]

\newcommand{\C}{\mathbb{C}}
\newcommand{\F}{\mathbb{F}}
\newcommand{\Q}{\mathbb{Q}}
\newcommand{\Z}{\mathbb{Z}}


\begin{document}

\maketitle

\begin{quote} 
    \center{\itshape Galois Theory studies symmetries among roots of polynomials.}
    \flushright{\small --- Professor (Spring 2023)}
\end{quote}

\tableofcontents


\section{Field Extensions}

\begin{theorem}
    \label{thm:field extension degrees multiply}
    If $F\to K$ and $K\to L$ are finite, then $[L:F]=[L:K][K:L]$.
\end{theorem}

\begin{proof}
    Let $\{\alpha_1,\dots,\alpha_n\}$ and $\{\beta_1,\dots,\beta_m\}$ be bases for $F\to K$ and $K\to L$
    respectively.
    We claim $S=\{\alpha_i\beta_j\}$ is an $F$-basis for $L$.
    It is immediate that $S$ spans $L$, so we show that $S$ is linearly independent.

    Suppose some linear combination of $S$ is zero, then factoring out by the $\alpha_i$ implies the coefficients of
    each group of $\alpha_i$ must be zero, but they are all linear combinations of the $\beta_j$, hence all
    coefficients must be zero.
\end{proof}

\begin{theorem}
    A field extension is finite if and only if it is algebraic and finitely generated.
\end{theorem}

\begin{proof}
    Suppose $F\to K$ is a field extension.
    It is trivial to show that
    \begin{enumerate}
        \item if $F\to K$ is not algebraic, then it is not finite, and
        \item if $F\to K$ is not finitely generated, then it is not finite.
    \end{enumerate}

    Suppose $F\to K$ is algebraic and finitely generated, and let $\{\alpha_1,\dots,\alpha_n\}$ be a basis for
    $F\to K$.
    Break the extension down by $F\to F(\alpha_1)\to F(\alpha_1,\alpha_2)\to\dots\to F(\alpha_1,\dots,\alpha_n)=K$, and
    see that each of these intermediate extensions are finite.
    Theorem $\ref{thm:field extension degrees multiply}$ asserts $F\to K$ is finite.
\end{proof}

\begin{corollary}
    Every composition of algebraic field extensions is algebraic.
\end{corollary}

\begin{proof}
    Suppose $F\to K$ and $K\to L$ are algebraic.
    Let $\alpha\in L$ with $m_{\alpha,K}=x^n+c_{n-1}x^{n-1}+\dots+c_0$, and construct $K^\prime=F(c_0,\dots,c_{n-1})$,
    which is algebraic and finitely generated, hence finite.
    But $K^\prime\to K^\prime(\alpha)$ is also finite, so $F\to K^\prime(\alpha)$ is finite, therefore $\alpha$ is
    algebraic over $F$.
\end{proof}

\begin{theorem}[Kronecker]
    \label{thm:Kronecker}
    If $F$ is a field and $f\in F[x]$ is non-constant, then there exists a finite $F\to K$ such that $f$ has a root in
    $K$.
\end{theorem}

\begin{proof}
    Without loss of generality, we may assume $f$ is irreducible.
    Define $K=F[x]/\langle f\rangle$, which is a field because $\langle f\rangle$ is maximal.
    See that $x+\langle f\rangle\in K$ is a root of $f$.
\end{proof}

\begin{theorem}
    A field $F$ is algebraically closed if and only if every algebraic $F\to K$ has $[K:F]=1$.
\end{theorem}

\begin{proof}
    The ``only if'' is trivial, so suppose every $F\to K$ has $[K:F]=1$.
    Let $f\in F[x]$, and Theorem \ref{thm:Kronecker} asserts there exists a finite $F\to K$ in which $f$ has a root.
    But $[K:F]=1$, so this root is in fact in $F$.
\end{proof}

\begin{theorem}
    Every field has an algebraic closure.
\end{theorem}

\begin{proof}
    Suppose $F$ is a field, and define $S$ to be the set of monic and irreducible polynomials in $F[x]$.
    Also construct $R=F[y_f\mid f\in S]$ and $I=\langle f(y_f)\mid f\in S\rangle$.

    We claim $1\notin I$.
    Towards a contradiction, suppose $1\in I$, so we can write $1=\sum_ia_if_i(y_{f_i})$ for $a_i\in R$ and $f_i\in S$.
    Repeating Theorem \ref{thm:Kronecker} for each $f_i$ generates a field extension in which there exist $\alpha_i$
    such that $f_i(\alpha_i)=0$ for all $i$, but now we have $1=\sum_ia_if_i(\alpha_i)=0$, a contradiction.

    Now we know $I$ is a proper ideal, so it is contained in some maximal ideal $M$.
    Define $F\to K=R/M$, and see that $y_i+\langle M\rangle\in K$ is a root of $f_i$, so we conclude that $K$ is an
    algebraic closure of $F$.
\end{proof}

\begin{theorem}[Isomorphism Extension]
    \label{thm:Isomorphism Extension}
    Let $F$ and $K$ be fields with isomorphism $\phi:F\to K$.
    If $F\to E$ is algebraic, then there exists an isomorphism $\psi$ between $E$ and a subfield of $\overline{K}$
    satisfying $\psi\vert_F=\phi$.
\end{theorem}

\begin{proof}
    Let $S$ be the set of $(E^\prime,\sigma)$ where $E^\prime$ is a field satisfying $F\subseteq E^\prime\subseteq E$
    and $\sigma$ an isomorphism from $E^\prime$ to a subfield of $\overline{K}$ satisfying $\sigma\vert_F=\phi$.
    Define a partial order on $S$ by $(E_1,\sigma_1)\leq (E_2,\sigma_2)$ if and only if $E_1\subseteq E_2$ and
    $\sigma_2\vert_{E_1}=\sigma_1$.
    We wish to apply Zorn's Lemma to $S$, so we note
    \begin{enumerate}
        \item that $(F,\phi)\in S$ implies $S$ is non-empty, and

        \item
            that every chain $(E_1,\sigma_1)\leq(E_2,\sigma_2)\leq\dots$ in $S$ is bounded above by
            $(E^\prime,\sigma)$, where $E^\prime=\bigcup_iE_i$ and $\sigma$ is defined by simply using whichever
            $\sigma_i$ is available, since they are all compatible.
    \end{enumerate}
    Therefore, there exists a maximal element $(M,\tau)\in S$, and we want to show $M=E$.
    Towards a contradiction, suppose $M\subsetneq E$ and choose $\alpha\in E\setminus M$ with minimal polynomial
    $m_{\alpha,M}=x^n+c_{n-1}x^{n-1}+\dots+c_0\in M[x]$.
    Define $L=\tau(M)$ and $f(x)=x^n+\tau(c_{n-1})x^{n-1}+\dots+\tau(c_0)\in L[x]$, and see that
    \[M(\alpha)\to M[x]/m_{\alpha,M}\to L[x]/f(x)\to L(\beta)\]
    is an isomorphism for $\beta\in\overline{L}=\overline{K}$ a root of $f(x)$.
    Moreover, this extends $\tau$, a contradiction.
\end{proof}

\begin{theorem}
    Let $F$ be a field and fix some $\overline{F}$.
    Every algebraic closure of $F$ is isomorphic to $\overline{F}$.
\end{theorem}

\begin{proof}
    Suppose $K$ is an algebraic closure of $F$.
    By Theorem \ref{thm:Isomorphism Extension}, there is an isomorphism between $K$ and a subfield $E$ of $\overline{F}$.
    But $E$ is algebraically closed, so $[\overline{F}:E]=1$, and therefore $E=\overline{F}$.
\end{proof}

\begin{theorem}[Symmetric Polynomials]
    Every symmetric polynomial can be written uniquely as a polynomial in the elementary symmetric polynomials.
\end{theorem}

\begin{theorem}[Algebra]
    The set of complex numbers is algebraically closed.
\end{theorem}


\section{Normal Extensions}

\begin{theorem}
    If $F\to K$ is algebraic, then it is equivalent to say
    \begin{enumerate}
        \item that $K$ is a splitting field,
        \item that every $\phi:K\to\overline{F}$ fixing $F$ induces an isomorphism on $K$, or
        \item that the minimal polynomial of every $\alpha\in K$ splits in $K[x]$.
    \end{enumerate}
\end{theorem}

\begin{proof}
    We first show $(i)\implies(ii)$.
    Suppose $K$ is a splitting field, and that $\phi:K\to\overline{F}$ fixes $F$.
    Let $\alpha\in K$, and see that $\phi(\alpha)$ must still be a root of $m_{\alpha,F}$, so therefore
    $\phi(\alpha)\in K$, and $\phi(K)\subseteq K$.
    On the other hand, since $\phi$ defines an injective endomap on the roots of $m_{\alpha,F}$, of which there are
    finitely many, it must in fact permute these roots.
    In particular, this means $\phi$ is bijective over $K$, so then $\phi(K)=K$.

    Now we show $(ii)\implies(iii)$.
    Let $\alpha\in K$, take $\beta\in\overline{F}$ a root of $m_{\alpha,F}$, and see that $\psi:F(\alpha)\to F(\beta)$
    generated by $\psi(\alpha)=\beta$ is an isomorphism.
    By Theorem \ref{thm:Isomorphism Extension}, there exists $\phi:K\to\overline{F}$ satisfying
    $\phi\vert_{F(\alpha)}=\psi$.
    But this means $\phi$ fixes $F$, so it induces an automorphism on $K$.
    In particular, this means $\phi(\alpha)=\beta\in K$, so every root of $m_{\alpha,F}$ is in $K$, which implies
    $m_{\alpha,F}$ splits in $K[x]$.

    Now for $(iii)\implies(i)$, see that $K$ is the splitting field of the minimal polynomials of every
    $\alpha\in K$.
\end{proof}


\end{document}
