\documentclass[
    parskip=half,
    toc=flat,
    toc=sectionentrydotfill,
]{scrartcl}  % KOMA-Script :o

\usepackage{amsmath}
\usepackage{amssymb}
\usepackage{amsthm}
\usepackage{cancel}
\usepackage{enumitem}
\usepackage{geometry}
\usepackage{xcolor}

\usepackage{hyperref}

\title{Galois Theory}
\subtitle{MATH 440}
\author{Steven Xia}

\changefontsizes{10pt}

% \geometry{paperheight=133.99mm, paperwidth=61.842mm} % estimated screen
% \geometry{paperheight=183.64mm, paperwidth=100.06mm, margin=2mm} % good
\geometry{paperheight=183.5mm, paperwidth=100mm, margin=2mm}

\linespread{1.236}

\definecolor{GithubSidebar}{HTML}{0d1117}
\pagecolor{GithubSidebar}
\color{white}

\theoremstyle{definition}
\newtheorem{definition}{Definition}[section]
\newtheorem{subdefinition}{Definition}[definition]

\theoremstyle{plain}
\newtheorem{theorem}{Theorem}[section]
\newtheorem{corollary}{Corollary}[theorem]
\newtheorem{proposition}{Proposition}[section]

\theoremstyle{remark}
\newtheorem{remark}{Remark}[section]

\newcommand{\Z}{\mathbb{Z}}
\renewcommand{\gcd}[1]{\textup{gcd}(#1)}
\newcommand{\lcm}[1]{\textup{lcm}(#1)}
\newcommand{\ord}[1]{\textup{ord}(#1)}

\setlist[enumerate,1]{label={(\roman*)}}

\begin{document}

\maketitle

\begin{minipage}{\textwidth} % Somehow makes this look better.
\begin{quote} 
    \center{\itshape Galois Theory is the study of symmetries among roots of polynomials.}
    \flushright{\small --- Professor (Spring 2023)}
\end{quote}
\end{minipage}

\tableofcontents


\section{Introduction}

\begin{definition}
    The \textbf{degree} of $K$ over $L$ is written $[K:L]$.
\end{definition}

\begin{proposition}
    Let $F\subseteq K$ and $K\subseteq L$ be field extensions.
    Then, $[L:F]=[L:K][K:F]$.
\end{proposition}

\begin{proof}
    We may assume $[L:K]$ and $[K:F]$ are finite.
    Then, $L$ has $K$-basis $\{a_1,\dots,a_k\}$ and $K$ has $F$-basis $\{b_1,\dots,b_f\}$.
    We can show $\{a_ib_j\}$ is an $F$-basis of $L$.
\end{proof}

\begin{definition}
    A field extension $F\subseteq K$ is \textbf{finite} if the degree of $K$
    over $F$ is finite.
\end{definition}

\begin{definition}
    A field extension $F\subseteq K$ is \textbf{finitely generated} if there
    exists a finite set $S$ such that $F(S)=K$.
\end{definition}

\begin{remark}
    \label{rem: finite implies finitely generated}
    A $F\subseteq K$ is finitely generated if it is finite.
\end{remark}

\begin{definition}
    For $F\subseteq K$, some $\alpha\in K$ is \textbf{algebraic} over $F$ if
    there exists a non-constant $f\in F[x]$ such that $f(\alpha)=0$.
\end{definition}

\begin{definition}
    The minimal polynoimial of $\alpha$ over $F$ is written $m_{\alpha,F}(\alpha)=0$.
    Then, the degree of $\alpha$ over $F$ is $\deg(m_{\alpha,F})$.
\end{definition}

\begin{definition}
    A $F\subseteq K$ is an algebraic field extension if every $\alpha\in K$ is algebraic.
\end{definition}

\begin{theorem}
    A $F\subseteq K$ is finite if and only if it is finitely generated and algebraic.
\end{theorem}

\begin{proof}
    Suppose $F\subseteq K$ is finite.
    We will show $K$ is algebraic over $F$ (finitely generated follows from
    Proposition \ref{rem: finite implies finitely generated}).
    Let $\alpha\in K$ be nonzero and see that $\alpha^0,\dots,\alpha^m\in K$ is
    linearly dependent if $m\geq\deg(m_{\alpha,F})=[K:F]$.

    Now, suppose $K=F(\alpha_1,\dots,\alpha_m)$ is algebraic and define
    $K_i=K_{i-1}(\alpha_i)$ with $K_0=F$.
    By an implicit induction on $i$, we see that $K_m=K$ is finite.
\end{proof}

\begin{corollary}
    Finite composition of algebraic and finitely generated field extensions
    are also finite.
\end{corollary}

\begin{definition}
    A field $F$ is \textbf{algebraically closed} if every non-constant
    $f\in F[x]$ has a root in $F$.
\end{definition}

\begin{remark}
    If $F$ is algebraically closed, every $f\in F[x]$ can be written as a
    product of linear factors.
\end{remark}

\begin{proposition}
    A field $F$ is algebraically closed if and only if every field extension
    $K$ of $F$ satisfies $[K:F]=1$.
\end{proposition}

\begin{proof}
    Assume $F$ is algebraically closed.
    Then, the minimal polynomial of every element over $F$ is linear, so any
    field extension over $F$ is of degree one.

    Now suppose every algebraic extension is of degree one.
    Consider some irreducible factor $f$ of a polynomial in $F[x]$ and the
    algebraic extension $F\to F[x]/\langle f\rangle$.
    Since the extension is of degree one, the degree of $f$ is also one.
\end{proof}

\begin{theorem}[Kronecker]
    Let $F$ be a field and $f\in F[x]$ be non-constant.
    There exists a finite extension $F\subseteq K$ such that $f$ has a root in
    $K$.
\end{theorem}

\begin{definition}
    An \textbf{algebraic closure} of a field $F$ is an algebraic extension
    $F\subseteq K$ such that $K$ is algebraically closed.
\end{definition}

\begin{theorem}
    Every field $F$ has an algebraic closure.
\end{theorem}

\begin{proof}
    Define $S$ as the set of monic and irreducible polynomials in $F[x]$,
    $R=F[y_f\mid f\in S]$, and $I=\langle f(y_f)\mid f\in S\rangle$.

    We claim that $I$ is a proper ideal, that is, $1\notin I$.
    Towards a contradiction, suppose $1\in I$.
    Then, we can write $1=\sum a_if_i(y_{f_i})$ for $f_i\in S$ and $a_i\in R$.
    However, repeating Kronecker's Theorem for each $f_i$ generates a field
    extension for which there exist $\alpha_i$ such that $f_i(\alpha_i)=0$ for
    all $i$, so we can plug these values into the sum to give $1=0$, a
    contradiction.

    Since every proper ideal is contained in a maximal ideal, there exists some
    $M\subseteq R$ such that $I\subseteq M$.
    Then, we define $F\subseteq K$ where $K=R/M$ as an algebraic field
    extension of $F$ generated by the $y_f$.
    Since $K$ contains a root to every irreducible polynomial, we conclude that
    it is an algebraic closure of $F$.
\end{proof}

\begin{theorem}
    All algebraic closures of a field are isomorphic.
\end{theorem}

\begin{definition}
    A \textbf{symmetric polynomial} $p\in F[x_1,\dots,x_n]$ satisfies
    $p(x_1,\dots,x_n)=p(x_{\sigma(1)},\dots,x_{\sigma(n)})$ for all
    $\sigma\in S_n$.
\end{definition}

\begin{definition}
    The elementary symmetric polynomials in $n$ variables are written $e_i$ for
    $1\leq i\leq n$ and are the sum of the $i$th degree monomials in the
    expansion of $\prod_{j=1}^n(1+x_j)$.
\end{definition}

\begin{theorem}[Symmetric Polynomials]
    All symmetric polynomials can be uniquely written as a polynomial in the
    elementary symmetric polynomials.
\end{theorem}

\begin{definition}
    The descriminant of a polynomial $f$ with roots $\alpha_1,\dots,\alpha_n$
    is $\Delta(f)=\prod_{i<j}(\alpha_i-\alpha_j)^2$.
\end{definition}

\begin{remark}
    The \textbf{discriminant} of $f$ is a symmetric polynomial in its roots,
    and the elementary symmetric polynomials are (up to negation) the
    coefficients of $f$.
\end{remark}

\begin{theorem}[Algebra]
    Every non-constant polynomial with complex coefficients has at least one
    complex root.
\end{theorem}


\section{Groups}


\begin{definition}
    The \textbf{automorphism group} of $K$, denoted $Aut(K)$, is the set of
    automorphisms of $K$.
\end{definition}

\begin{definition}
    The \textbf{Galois group} of a field extension $F\subseteq K$, denoted
    $Gal(K/F)$, is the set of automorphisms of $K$ such that $F$ is fixed
    pointwise.
\end{definition}

\begin{definition}
    A \textbf{left action} of a group $G$ on a set $X$ is a map
    $G\times X\to X$ written $(g,x)\mapsto g.x$ such that $e\in G$ satisfies
    $e.x=x$ for all $x$ and $g.(h.x)=(gh).x$ for all $g,h\in G$.
\end{definition}

\begin{definition}
    The \textbf{standard left action} of $H\leq G$ on $G$ is the left action
    defined $(h,g)\mapsto hg$.
\end{definition}

\begin{definition}
    The \textbf{conjugation action} of $H\leq G$ on $G$ is the left action
    defined $(h,g)\mapsto hgh^{-1}$.
\end{definition}

\begin{definition}
    The \textbf{orbit} of $x\in X$ under group action $G$ is defined
    $G.x=\{g.x:g\in G\}$.
\end{definition}

\begin{theorem}
    The orbits under an action form a partition.
\end{theorem}

% \begin{proof}
%     Let $G$ act on $X$.
%     Suppose $y\in G.x$ by $y=g.x$ where $g\in G$ and $x,y\in X$.
%     Then, $h.x=h.(g^{-1}.y)$ for all $h\in G$.
% \end{proof}

\begin{definition}
    The \textbf{stabilizers} of $x\in X$ under group action $G$ is defined
    $G_x=\{g\in G:g.x=x\}$.
\end{definition}

\begin{theorem}
    \label{thm:stabilizers are groups}
    Every stabilizer forms a group.
\end{theorem}

\begin{definition}
    For groups $H\leq G$ and $g\in G$, a \textbf{left coset} of $H$ in $G$ is
    defined $gH=\{gh:h\in H\}$.
    We write $G/H$ to denote the set of left cosets of $H$ in $G$.
\end{definition}

\begin{definition}
    The \textbf{index} of $H$ in $G$ is the number of left cosets of $H$ in
    $G$.
    We write $[G:H]$ to denote this value.
\end{definition}

\begin{definition}
    Orbits under a conjugation action $H\leq G$ are called
    \textbf{conjugacy classes} under conjugation by $H$.
\end{definition}

\begin{theorem}[Orbit-stabilizer]
    Let $H\leq G$ be groups.
    For all $x\in G$, there is a bijection $G.x\to G/G_x$.
\end{theorem}

\begin{proof}
    Define $\phi:G\to G.x$ where $\phi(g)=g.x$, a surjective map.
    We see that $\phi(g)=\phi(h)\iff g.x=h.x\iff g^{-1}h\in G_x$.
    By Theorem $\ref{thm:stabilizers are groups}$, it follows that $gG_x=hG_x$.
    Therefore, the map $gG_x\mapsto g.x$ is a bijection.
\end{proof}


\end{document}
