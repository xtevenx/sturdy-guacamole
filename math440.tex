\documentclass[
    parskip=half,
    toc=flat,
    toc=sectionentrydotfill,
]{scrartcl}  % KOMA-Script :o

\usepackage{amsmath}
\usepackage{amssymb}
\usepackage{amsthm}
\usepackage{cancel}
\usepackage{enumitem}
\usepackage{geometry}
\usepackage{xcolor}

\usepackage{hyperref}

\title{Galois Theory}
\subtitle{MATH 440}
\author{Steven Xia}

\changefontsizes{10pt}

% \geometry{paperheight=133.99mm, paperwidth=61.842mm} % estimated screen
% \geometry{paperheight=183.64mm, paperwidth=100.06mm, margin=2mm} % good
\geometry{paperheight=183.5mm, paperwidth=100mm, margin=2mm}

\linespread{1.236}

\definecolor{GithubSidebar}{HTML}{0d1117}
\pagecolor{GithubSidebar}
\color{white}

\theoremstyle{definition}
\newtheorem{definition}{Definition}[section]
\newtheorem{subdefinition}{Definition}[definition]

\theoremstyle{plain}
\newtheorem{theorem}{Theorem}[section]
\newtheorem{corollary}{Corollary}[theorem]
\newtheorem{proposition}{Proposition}[section]

\theoremstyle{remark}
\newtheorem{remark}{Remark}[section]

\newcommand{\Z}{\mathbb{Z}}
\renewcommand{\gcd}[1]{\textup{gcd}(#1)}
\newcommand{\lcm}[1]{\textup{lcm}(#1)}
\newcommand{\ord}[1]{\textup{ord}(#1)}

\setlist[enumerate,1]{label={(\roman*)}}

\begin{document}

\maketitle

\begin{minipage}{\textwidth} % Somehow makes this look better.
\begin{quote} 
    \center{\itshape Galois Theory is the study of symmetries among roots of polynomials.}
    \flushright{\small --- Professor (Spring 2023)}
\end{quote}
\end{minipage}

\tableofcontents


\section{Introduction}

\begin{definition}
    The \textbf{degree} of $K$ over $L$ is written $[K:L]$.
\end{definition}

\begin{proposition}
    Let $F\subseteq K$ and $K\subseteq L$ be field extensions.
    Then, $[L:F]=[L:K][K:F]$.
\end{proposition}

\begin{proof}
    We may assume $[L:K]$ and $[K:F]$ are finite.
    Then, $L$ has $K$-basis $\{a_1,\dots,a_k\}$ and $K$ has $F$-basis $\{b_1,\dots,b_f\}$.
    We can show $\{a_ib_j\}$ is an $F$-basis of $L$.
\end{proof}

\begin{definition}
    A field extension $F\subseteq K$ is \textbf{finite} if the degree of $K$
    over $F$ is finite.
\end{definition}

\begin{definition}
    A field extension $F\subseteq K$ is \textbf{finitely generated} if there
    exists a finite set $S$ such that $F(S)=K$.
\end{definition}

\begin{proposition}
    \label{prop: finite implies finitely generated}
    A $F\subseteq K$ is finitely generated if it is finite.
\end{proposition}

\begin{definition}
    For $F\subseteq K$, some $\alpha\in K$ is \textbf{algebraic} over $F$ if
    there exists a non-constant $f\in F[x]$ such that $f(\alpha)=0$.
\end{definition}

\begin{definition}
    The minimal polynoimial of $\alpha$ over $F$ is written $m_{\alpha,F}(\alpha)=0$.
    Then, the degree of $\alpha$ over $F$ is $\deg(m_{\alpha,F})$.
\end{definition}

\begin{definition}
    A $F\subseteq K$ is an algebraic field extension if every $\alpha\in K$ is algebraic.
\end{definition}

\begin{theorem}
    A $F\subseteq K$ is finite if and only if it is finitely generated and algebraic.
\end{theorem}

\begin{proof}
    Suppose $F\subseteq K$ is finite.
    We will show $K$ is algebraic over $F$ (finitely generated follows from
    Proposition \ref{prop: finite implies finitely generated}).
    Let $\alpha\in K$ be nonzero and see that $\alpha^0,\dots,\alpha^m\in K$ is
    linearly dependent if $m\geq\deg(m_{\alpha,F})=[K:F]$.

    Now, suppose $K=F(\alpha_1,\dots,\alpha_m)$ is algebraic and define
    $K_i=K_{i-1}(\alpha_i)$ with $K_0=F$.
    By an implicit induction on $i$, we see that $K_m=K$ is finite.
\end{proof}


\end{document}
