\documentclass[
    parskip=half,
    toc=flat,
    toc=sectionentrydotfill,
]{scrartcl}  % KOMA-Script :o

\usepackage{amsmath}
\usepackage{amssymb}
\usepackage{amsthm}
\usepackage{cancel}
\usepackage{enumitem}
\usepackage{geometry}
\usepackage{xcolor}

\usepackage{hyperref}

\title{Algebra III: Groups}
\subtitle{MATH 341}
\author{Steven Xia}

\changefontsizes{10pt}

\geometry{paperheight=183.5mm, paperwidth=100mm, margin=2mm}

\linespread{1.236}

\definecolor{GithubSidebar}{HTML}{0d1117}
\pagecolor{GithubSidebar}
\color{white}

\theoremstyle{definition}
\newtheorem{definition}{Definition}[section]
\newtheorem{subdefinition}{Definition}[definition]

\theoremstyle{plain}
\newtheorem{theorem}{Theorem}[section]
\newtheorem{corollary}{Corollary}[theorem]
\newtheorem{proposition}{Proposition}[section]

\theoremstyle{remark}
\newtheorem{remark}{Remark}[section]

\newcommand{\Z}{\mathbb{Z}}
\renewcommand{\gcd}[1]{\textup{gcd}(#1)}
\newcommand{\lcm}[1]{\textup{lcm}(#1)}
\newcommand{\ord}[1]{\textup{ord}(#1)}

\setlist[enumerate,1]{label={(\roman*)}}

\begin{document}
\maketitle
\tableofcontents

\section{Assorted Introductions}

\begin{center}
    \textbf{Author's Remark}\\
    This section is weird because most of the material was already covered
    either in MATH 340 or MATH 440.
    There is missing material, and I was simply too lazy to add it.
\end{center}

\begin{theorem}[Internal Characterization]
    For $G_1,G_2\subseteq G$ groups, $G\cong G_1\times G_2$ if and only if all
    the following apply:
    \begin{enumerate}[nosep]
        \item $G=\{g_1g_2:g_1\in G_1,g_2\in G_2\}$,
        \item $G_1\cap G_2=\{e_G\}$, and
        \item $g_1g_2=g_2g_1$ for all $g_1\in G_1$ and $g_2\in G_2$.
    \end{enumerate}
\end{theorem}

\begin{proof}
    Tedious rule checking.
\end{proof}

\begin{theorem}[Cayley]
    Every finite group of order $n$ is isomorphic to some subgroup of $S_n$.
\end{theorem}

\begin{proof}
    Let $G$ be a finite group of order $n$.
    Define $\phi:G\to S_n$ as $\phi(g)=\sigma_g$, where $\sigma_g(h)=gh$, an
    isomorphism.
\end{proof}

\begin{theorem}[Lagrange]
    Let $H\subseteq G$ be groups (not necessarily finite).
    Then, $|G|=[G:H]|H|$.
\end{theorem}

\begin{proof}
    We prove only for the finite case, by seeing that cosets partition the
    group, and that all cosets are of the same size.
\end{proof}

\begin{theorem}[Cauchy]
    Let $G$ be a finite group of order $n$.
    If a prime $p$ divides $n$, there exists an element of order $p$.
\end{theorem}

\begin{proof}
    Define $X=\{(x_1,\dots,x_p)\in G^p:x_1\cdots x_p=e\}$ and see that $x_p$
    is determined entirely by the choices of $x_1,\dots,x_{p-1}$.
    Since $x_1,\dots,x_{p-1}$ can be chosen arbitrarily, $|X|=n^{p-1}$.

    Let $C_p$ act on $X$ by cyclic permutation of the $p$-tuple.
    Since stabilizers are subgroups, the orbit-stabilizer theorem says that all
    orbits of $X$ are size either $1$ or $p$.
    We note an orbit of some $(x_1,\dots,x_p)\in X$ is size $1$ if and only if
    $x_1=\dots=x_p$, id est, $x_1$ is of order $p$ or $x_1=e$.
    Finally, since $|X|$ is a multiple of $p$, the class equation says there
    must be at least $p$ elements of with an orbit of size $1$, hence $p-1$
    elements of order $p$.
\end{proof}

\begin{theorem}
    Let $H\subseteq G$ be a normal subgroup.
    Then, the quotient set $G/H$ has a group structure.
\end{theorem}

\begin{proof}
    Tedious rule checking.
\end{proof}


\end{document}
