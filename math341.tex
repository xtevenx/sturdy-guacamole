\documentclass[
    parskip=half,
    toc=flat,
    toc=sectionentrydotfill,
]{scrartcl}  % KOMA-Script :o

\usepackage{amsmath}
\usepackage{amssymb}
\usepackage{amsthm}

\usepackage{fontspec}
\usepackage{unicode-math}
\setmainfont{STIX Two Text}
\setmathfont{STIX Two Math}

\usepackage{geometry}
\geometry{paperheight=183.5mm, paperwidth=100mm, margin=2mm}

\usepackage{xcolor}
\definecolor{GithubSidebar}{HTML}{0d1117}
\pagecolor{GithubSidebar}
\color{white}

\usepackage{enumitem}
\setlist[enumerate,1]{label={(\roman*)}}

\usepackage{hyperref}

\theoremstyle{definition}
\newtheorem{definition}{Definition}[section]

\theoremstyle{plain}
\newtheorem{theorem}{Theorem}[section]
\newtheorem{corollary}{Corollary}[theorem]

\theoremstyle{remark}
\newtheorem{remark}{Remark}[section]

\title{Algebra III: Groups}
\subtitle{MATH 341}
\author{Steven Xia}

\changefontsizes{10pt}
\linespread{1.236}

\begin{document}
\maketitle
\tableofcontents


\section{Assorted Introductions}

\begin{center}
    \textbf{Author's Remark}\\
    This section is weird because most of the material was already covered either in MATH 340 or MATH 440.
    That it to say, there is material that I am too lazy to add.
\end{center}

\begin{theorem}[Internal Characterization]
    For $G_1,G_2\subseteq G$ groups, $G\cong G_1\times G_2$ if and only if all the following apply:
    \begin{enumerate}[nosep]
        \item $G=\{g_1g_2:g_1\in G_1,g_2\in G_2\}$,
        \item $G_1\cap G_2=\{e_G\}$, and
        \item $g_1g_2=g_2g_1$ for all $g_1\in G_1$ and $g_2\in G_2$.
    \end{enumerate}
\end{theorem}

\begin{proof}
    Tedious rule checking.
\end{proof}

\begin{theorem}[Cayley]
    Every finite group of order $n$ is isomorphic to some subgroup of $S_n$.
\end{theorem}

\begin{proof}
    Let $G$ be a finite group of order $n$.
    Define $\phi:G\to S_n$ as $\phi(g)=\sigma_g$ where $\sigma_g(h)=gh$, an isomorphism.
\end{proof}

\begin{theorem}[Lagrange]
    Let $H\subseteq G$ be groups (not necessarily finite).
    Then, $|G|=[G:H]|H|$ (with cardinality if infinite).
\end{theorem}

\begin{proof}
    We prove only for the finite case, by seeing that cosets partition the group, and that all cosets are of the same
    size.
\end{proof}

\begin{theorem}[Cauchy]
    Let $G$ be a finite group of order $n$.
    If a prime $p$ divides $n$, there exists an element of order $p$.
\end{theorem}

\begin{proof}
    Define $X=\{(x_1,\dots,x_p)\in G^p:x_1\cdots x_p=e\}$ and see that $x_p$ is determined entirely by the choices of
    $x_1,\dots,x_{p-1}$.
    Since $x_1,\dots,x_{p-1}$ can be chosen arbitrarily, $|X|=n^{p-1}$.

    Let $\mathbb{Z}_p$ act on $X$ by cyclic permutation of the $p$-tuple.
    Since stabilizers are subgroups, the orbit-stabilizer theorem says that all orbits of $X$ are size either $1$ or
    $p$.
    We note an orbit of some $(x_1,\dots,x_p)\in X$ is size $1$ if and only if $x_1=\dots=x_p$, id est, $x_1$ is of
    order $p$ or $x_1=e$.
    Finally, since $|X|$ is a multiple of $p$, the class equation says there must be at least $p$ elements of with an
    orbit of size $1$, hence $p-1$ elements of order $p$.
\end{proof}

\begin{theorem}
    Let $H\subseteq G$ be a normal subgroup.
    Then, the quotient set $G/H$ has a group structure.
\end{theorem}

\begin{proof}
    Tedious rule checking.
\end{proof}

\begin{theorem}
    A group $H\subseteq G$ is normal if and only if $gHg^{-1}\subseteq H$ for all $g\in G$.
\end{theorem}

\begin{proof}
    The ``only if'' is trivial, so we prove the ``if'' direction by showing $gHg^{-1}\supseteq H$.
    We see $gHg^{-1}\subseteq H\implies g^{-1}Hg\subseteq H$.
    For some $ghg^{-1}\in H$, we have $ghg^{-1}=h^\prime\iff h=g^{-1}h^\prime g$ for some $h^\prime\in H$.
    But then $g^{-1}h^\prime g\in g^{-1}Hg$ so $h=gh^\prime g\in H$.
\end{proof}

\begin{theorem}
    Subgroups of index 2 are normal.
\end{theorem}

\begin{proof}
    Let $H\subseteq G$ be a subgroup of index 2.
    Then, the left cosets of $H$ are $H,gH$ and the right cosets of $H$ are $H,Hg$, so $gH=Hg$.
\end{proof}

\begin{theorem}[First Isomorphism]
    Let $\phi:G\to H$ be a surjective homomorphism.
    Then, $H\cong G/\ker(\phi)$.
\end{theorem}

\begin{proof}
    Tedious rule checking.
\end{proof}

\begin{corollary}
    Let $\phi:G\to H$ be a homomorphism.
    Cosets of $\ker(\phi)$ contain all values mapping to some element in the codomain.
\end{corollary}

\begin{remark}
    Normal subgroups of a group are exactly all possible kernels of homomorphisms from that group.
\end{remark}


\section{Alternating Groups}

\begin{definition}
    A \textbf{transposition} in a symmetric group is a cycle with support size 2.
\end{definition}

\begin{remark}
    Every permutation can be written (not uniquely) as a product of transpositions.
\end{remark}

\begin{theorem}
    For some permutation $\sigma$, all writings of $\sigma$ as a product of transpositions has the same parity of the
    number of factors.
\end{theorem}

\begin{proof}
    Let $S_n$ act on $f=\prod_{i<j}(x_i-x_j)\in F[x_1,\dots,x_n]$ by permuting the $x$'s, and see that every
    transposition negates $f$.
\end{proof}

\begin{remark}
    The parity of a permutation $\sigma$ is the number of transpositions in a 2-cycle factorization of $\sigma$.
\end{remark}

\begin{definition}
    The \textbf{alternating group} on $n$ elements, $A_n$, is the set of even permutations.
\end{definition}

\begin{definition}
    The \textbf{cycle structure} of a permutation $\sigma$ is the list of cycle lengths in $\sigma$ sorted in
    non-increasing order.
\end{definition}

\begin{theorem}[Conjugation]
    Every conjugate of a permutation has the same cycle structure.
\end{theorem}

\begin{proof}
    Let $\alpha\in S_n$ and see $\alpha(a_1\cdots a_k)\alpha^{-1}=(\alpha(a_1)\cdots\alpha(a_k))$.
\end{proof}

\begin{theorem}
    Every pair of permutations of the same cycle structure is conjugates in $S_n$.
\end{theorem}

\begin{definition}
    A group is \textbf{simple} if the only normal subgroups are the trivial group or itself.
\end{definition}

\begin{theorem}
    The group $A_n$ is simple if and only if $n\neq 4$.
\end{theorem}

\begin{proof}[Proof sketch]
    Let $H\subseteq A_n$ be proper normal, and choose $\alpha\in H$ of minimal support.
    Claim $\alpha$ must be a cycle of length 3, so normality forces $H$ to have all cycles of length 3, which generates
    $A_4$.
\end{proof}


\section{Sylow Theorems}


\begin{definition}
    For prime $p$, a \textbf{$p$-group} is one which only contains elements of order a power of $p$.
\end{definition}

\begin{theorem}
    A finite group is a $p$-group if and only if it is of order a power of $p$.
\end{theorem}

\begin{theorem}
    Let $G$ be an abelian group with $|G|=p^nm$ where $\gcd(p,m)=1$.
    Defining $P$ as all elements of order a power of $p$ and $M$ as all other elements, we see $G\cong P\times M$.
\end{theorem}


\end{document}
