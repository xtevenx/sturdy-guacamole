\documentclass[parskip=half]{scrartcl}  % KOMA-Script :o

\usepackage{amsmath}
\usepackage{amssymb}
\usepackage{amsthm}
\usepackage{cancel}
\usepackage{geometry}
\usepackage{xcolor}

\usepackage{hyperref}

\title{Algebra II: Rings and Fields}
\subtitle{MATH 340}
\author{Steven Xia}

\changefontsizes{10pt}

% \geometry{paperheight=133.99mm, paperwidth=61.842mm} % estimated screen
% \geometry{paperheight=183.64mm, paperwidth=100.06mm, margin=2mm} % good
\geometry{paperheight=183.5mm, paperwidth=100mm, margin=2mm}

\pagenumbering{gobble}
\linespread{1.236}

\definecolor{GithubHeader}{HTML}{161b22}
\pagecolor{GithubHeader}
% \definecolor{GithubSidebar}{HTML}{0d1117}
% \pagecolor{GithubSidebar}
\color{white}

\theoremstyle{definition}
\newtheorem{definition}{Definition}[section]
\newtheorem{subdefinition}{Definition}[definition]

\theoremstyle{plain}
\newtheorem{theorem}{Theorem}[definition]
\newtheorem{corollary}{Corollary}[definition]
\newtheorem{proposition}{Proposition}[definition]

\theoremstyle{remark}
\newtheorem{remark}{Remark}[definition]

\begin{document}
\maketitle


\section{Fundamental Definitions}


\begin{definition}
    A \textbf{group} is a pairing of a set and a binary operation such that the
    operation is associative, and each element of the set has both an inverse
    and an identity.
\end{definition}

\begin{remark}
    Considering an identity element is defined both as a left and a right
    identity, it can be proven that it is unique for the group.
\end{remark}

\begin{remark}
    Similarly, we can prove that the inverse is unique for every element of the set.
\end{remark}

\begin{subdefinition}
    A \textbf{commutative group} is a group where the binary operation is
    commutative.
\end{subdefinition}

\begin{definition}
    A \textbf{ring} is a commutative group with another operation defined such
    that the two operations are similar to ``addition'' and ``multiplication''
    of the integers.
    The multiplication operation must be associative and distributive.
\end{definition}

\begin{subdefinition}
    A \textbf{commutative ring} is a ring where multiplication is commutative.
\end{subdefinition}

\begin{subdefinition}
    We say that a ring has \textbf{unity} if there is a multiplicative identity.
\end{subdefinition}

\begin{remark}
    \label{rem:zero multiply}
    For any ring $R$ with additive identity 0, it can be proven that $0a=0$ for
    all $a\in R$.
\end{remark}

\begin{remark}
    Using Remark \ref{rem:zero multiply}, it can be proven that
    $-ab=(-a)b=a(-b)$ for all $a,b\in R$.
\end{remark}

\begin{subdefinition}
    For some non-zero $a,b\in R$, we say $a$ and $b$ are \textbf{zero divisors}
    if $ab=0$.
\end{subdefinition}

\begin{remark}
    \label{rem:zero divisor}
    For some non-zero $a\in R$, it can be proven that $a$ is a left zero
    divisor if and only if there exists non-zero $b,c\in R$ such that $b\neq c$
    and $ab=ac$.
\end{remark}

\begin{remark}
    It follows from Remark \ref{rem:zero divisor}, that if a ring $R$ does
    \textit{not} have any zero divisors, then $ab=ac\implies b=c$ for all
    $a,b,c\in R$ and $a\neq 0$.
\end{remark}

\begin{subdefinition}
    A \textbf{unit} in a ring with unity is an element which has a
    multiplicative inverse.
\end{subdefinition}

\begin{remark}
    A unit cannot be a zero-divisor.
\end{remark}

\begin{definition}
    An \textbf{integral domain} is a commutative ring with unity and no zero
    divisors.
\end{definition}

\begin{definition}
    A \textbf{field} is a commutative ring where every non-zero element is a
    unit, and the additive and multiplicative identities are not equal.
\end{definition}


\section{Basic Proofs}


\begin{definition}
    The \textbf{characteristic} of a ring is the least positive integer $c$
    such that $\underbrace{1+1+\cdots+1}_{c\text{ times}}=0$.
    If this number does not exist, define the characteristic to be 0.
\end{definition}

\begin{theorem}
    \label{thm:composite characteristic implies zero divisor}
    If the characteristic of a ring is composite, it must have zero divisors.
\end{theorem}

\begin{proof}
    Let $c$ be the characteristic of some ring where there exists positive
    integers $m,n$ such that $c=mn$ and $m,n<c$.
    Consider, using the distributivity of multiplication, that
    $(\underbrace{1+1+\cdots+1}_{m\text{ times}})(\underbrace{1+1+\cdots+1}_{n\text{ times}})=0$.
\end{proof}

\begin{theorem}[Euler's Theorem]
    Let $R^*$ be the finite set of the units in a ring.
    For all $a\in R^*$, $a^{|R^*|}=1$.
\end{theorem}

\begin{proof}
    We have $R^*=\{r_1,\dots,r_n\}=\{ar_1,\dots,ar_n\}$ since $a$ is not a
    zero-divisor (it is a unit) so $ar_i\neq ar_j$.
    Then, $r_1\cdots r_n=(ar_1)\cdots (ar_n)=a^n(r_1\cdots r_n)\implies a^n=1$.
\end{proof}

\begin{theorem}
    \label{thm:finite ring with unity}
    For a finite ring with unity, any element is either 0, a zero divisor, or a
    unit.
\end{theorem}

\begin{proof}
    For an element $r$ that is not zero or a zero divisor, we have the
    following set of non-zero elements $\{r,r^2,\dots\}$.
    Since the ring is finite, we have $r^{e_1}=r^{e_2}$ for some $e_1<e_2$.
    Then, $r^{e_1}=r^{e_2}=r^{e_1}r^{e_2-e_1}\implies r^{e_2-e_1}=1$.
    Therefore, $r\cdot r^{e_2-e_1-1}=1$.
\end{proof}

\begin{remark}
    It follows from Theorem \ref{thm:finite ring with unity} that every finite
    integral domain is a field.
\end{remark}


\section{Unique Factorization}


\begin{definition}
    A \textbf{quadratic ring} extension $R[\gamma]$ of some ring $R$ is created
    by adding an element $\gamma$ to $R$ such that $\gamma^2=c$ for some
    $c\in R$ and $\gamma\notin R$.
\end{definition}

\begin{remark}
    Elements in $R[\gamma]$ are denoted $a+\gamma b$ for $a,b\in R$.
    This means elements in $R[\gamma]$ can be seen as elements in $R\times R$.
\end{remark}

\begin{theorem}
    \label{thm:norm map}
    The norm map\footnotemark $N:R[\gamma]\to R$ is defined as
    $N(a+\gamma b)=a^2-cb^2$ and has the property that $N(a+\gamma b)$ is a
    unit in $R$ if and only if $a+\gamma b$ is a unit in $R[\gamma]$.
    \footnotetext{We have not yet formally defined a \textit{norm map}.}
\end{theorem}

\begin{proof}
    We see that $N(a+\gamma b)^{-1}$ exists if and only if $N(a+\gamma b)$ is a unit.
    Then, $(a+\gamma b)(a-\gamma b)=N(a+\gamma b)$ so
    $(a+\gamma b)\left[(a-\gamma b)N(a+\gamma b)^{-1}\right]=1$.
\end{proof}

\begin{remark}
    Theorem \ref{thm:norm map} shows the quadratic ring extension of any field
    or integral domain maintains that status.
\end{remark}

\begin{definition}
    An element in an integral domain is called \textbf{irreducible} if it
    cannot be written as a product of two non-units.
\end{definition}

\begin{definition}
    Elements $a,b$ in an integral domain $R$ are called \textbf{associates} if
    there exists a unit $u\in R$ such that $a=ub$.
\end{definition}

\begin{definition}
    \label{def:unique factorization}
    An integral domain has \textbf{unique factorization} if every element can
    be written as a product of irreducibles which are unique up to order and
    associates.
\end{definition}

\begin{theorem}
    \label{thm:unique factorization}
    An integral domain $R$ has unique factorization if all irreducible elements
    are prime.
\end{theorem}

\begin{proof}
    Let $x=a_1\cdots a_n=b_1\cdots b_m$.
    Since $a_1$ is prime, we know that it divides one of $b_i$.
    Without loss of generality, let $b_1=ca_1$.
    However, since $b_1$ is irreducible and $a_1\neq 1$, we have $c=1$.
    Then, we can repeat this process on $a_2\cdots a_n=b_2\cdots b_m$.
\end{proof}

\begin{definition}
    A \textbf{polynomial ring} $R[x]$ of some ring $R$ is created by using
    polynomials of the variable $x$ using coefficients from $R$.
\end{definition}

\begin{remark}
    For some field $F$, Euclidean division works on $F[x]$ because all non-zero
    coefficients are units.
    It then follows that irreducible elements are prime, so unique factorization
    exists in $F[x]$.
\end{remark}

\begin{theorem}[Fundamental Theorem of Algebra]
    \label{thm:fundamental theorem of algebra}
    The only irreducible polynomials in $\mathbb{C}[x]$ are linear.
\end{theorem}

\begin{remark}
    It follows from Theorem \ref{thm:fundamental theorem of algebra} that the
    only irreducible polynomials in $\mathbb{R}[x]$ are linear or quadratic.
    This can be proven using $\mathbb{R}[x]\subset\mathbb{C}[x]$ and that
    multiplying some linear $f(x)\in\mathbb{C}[x]$ with its conjugate results
    in some $g(x)\in\mathbb{R}[x]$ with $\deg(g(x))=2$.
\end{remark}

\begin{definition}
    A subring $I$ of ring $R$ is called an \textbf{ideal} if for all
    $a\in I$ and $r\in R$, $ra,ar\in I$.
\end{definition}

\begin{definition}
    For a commutative ring $R$ and $a\in R$, a \textbf{principal ideal}
    generated by $a$ is defined as $aR=\{ar:r\in R\}$.
    For $a,b\in R$, we can also generate $(a,b)R=\{xa+yb:x,y\in R\}$.
\end{definition}

\begin{remark}
    For $a\in R$ with integral domain $R$, $aR=1R=R$ if and only if $a$ is a
    unit.
\end{remark}

\begin{remark}
    For $a,b\in R$, $b\mid a\implies aR\subseteq bR$.
    Furthermore, $aR=bR$ if and only if $a$ and $b$ are associates.
\end{remark}

\begin{remark}
    For $a,b\in R$, if $a$ is irreducible and $b\mid a$, then
    $aR\subseteq bR\subseteq R$ so either $aR=bR$ or $bR=R$.
    Therefore, $aR$ is not properly contained in any other principal ideal.
    Also, if $a$ is not irreducible and $b$ is not a unit, then
    $aR\subset bR\subset R$.
\end{remark}

\begin{theorem}
    If an element $a\in R$ cannot be written as a finite product of
    irreducibles, then $R$ has an infinite ascending chain of principal ideals.
\end{theorem}

\begin{proof}
    Assume that $a$ cannot be written as a finite product of irreducibles.
    Then, $a=r_1a_1=r_1r_2a_2=\dots$ for non-units $r_i,a_i$ and reducible $a_i$.
    This implies $aR\subset a_1R\subset a_2R\subset\cdots$.
\end{proof}

\begin{remark}
    This tells us that every element in $\mathbb{N}$ has a factorization into
    irreducibles since every proper divisor is ``smaller'' so there cannot be
    an infinite chain.
\end{remark}

\begin{definition}
    An integral domain $R$ is a \textbf{principal ideal domain} if every ideal
    in $R$ is a principal ideal.
\end{definition}

\begin{proposition}
    \label{thm:integers is principal ideal domain}
    The ring $\mathbb{Z}$ is a principal ideal domain.
\end{proposition}

\begin{proof}
    If $I=\{0\}$, then $I=0\mathbb{Z}$.
    Therefore, we prove with $I\neq\{0\}$.
    Then, there exists a positive element in $I$.
    Let $a$ be the the least positive element in $I$ and we claim that
    $I=a\mathbb{Z}$.

    Let $b\in I$ be some other element in $I$.
    Then we have $b = qa + r$ for $0\leq r<a$.
    This also means $b-qa=r$ so $r\in I$.
    However, by the minimality of $a$, this implies $r=0$ so $b$ is a multiple
    of $a$ and $b\in a\mathbb{Z}$.
\end{proof}

\begin{corollary}
    For field $F$, $F[x]$ is a principal ideal domain.
\end{corollary}

\begin{theorem}
    For a principal ideal domain, every ascending chain of ideals stabilizes.
\end{theorem}

\begin{proof}
    Let $I_1\subseteq I_2\subseteq\cdots$ be an ascending chain of ideals in a
    principal ideal domain $R$.
    Then, $\bigcup_{i=1}^\infty I_i$ is a principal ideal $aR$.
    For some $j$, $a\in I_j$ so $aR=I_j=I_{j+1}=\cdots$.
\end{proof}

\begin{definition}
    Let $I,J$ be ideals of $R$. Then, $I+J$ is the smallest ideal which contains
    both $I$ and $J$.
    Therefore, $I+J=\{a+b:a\in I\text{ and }b\in J\}$.
\end{definition}

\begin{remark}
    Since $\mathbb{Z}$ is a principal ideal domain, $a\mathbb{Z}+b\mathbb{Z}=d\mathbb{Z}$.
    Then, $d\mathbb{Z}=\{xa+yb:x,y\in\mathbb{Z}\}$.
    Therefore, $d=\gcd(a,b)$ since it is the least positive element (by proof
    of Theorem \ref{thm:integers is principal ideal domain}).
\end{remark}

\begin{definition}
    An ideal $I$ of ring $R$ is a \textbf{prime ideal} if $ab\in I$ implies
    $a\in I$ or $b\in I$ for all $a,b\in R$.
\end{definition}

\begin{remark}
    An element $p\in R$ is prime if and only if $pR$ is prime.
    Why?
\end{remark}

\begin{remark}
    Not all prime ideals are principal (eg. $(x,y)\subset\mathbb{Q}[x,y]$).
\end{remark}

\begin{definition}
    An ideal $I$ in ring $R$ is called \textbf{maximal} if for any ideal
    $J\subseteq R$ where $I\subseteq J\subseteq R$, it follows that $I=J$ or
    $J=R$.
\end{definition}

\begin{remark}
    In a principal ideal domain, the principal ideal generated by an
    irreducible element is maximal.
\end{remark}

\begin{theorem}
    In an integral domain, maximal ideals are prime.
\end{theorem}

\begin{proof}
    Let $I$ be a maximal ideal of ring $R$ with $bc\in I$.
    Suppose $b\notin I$.
    Then, we have $I\subsetneq I+bR\subseteq R$ so, by the maximality of $I$,
    $I+bR=R$.
    This also means that $1\in I+bR$ so $1=a+br$ for $a\in I$ and $r\in R$.
    Multiplying through by $c$, this gives us $c=ac+bcr\in I$ since $a,bc\in I$.
\end{proof}

\begin{remark}
    For a principal ideal domain $R$, this gives us that $a\in R$ is
    irreducible implies $aR$ is maximal implies $aR$ is prime implies a is prime.
    Therefore, by Theorem \ref{thm:unique factorization}, every principal ideal
    domain has unique factorization.
\end{remark}

\begin{definition}
    A ring is a unique factorization domain if every non-zero non-unit can be
    written uniquely as a product of irreducible elements, up to order and
    associates.
    Duplicate of Definition \ref{def:unique factorization}; don't ask why.
\end{definition}

\begin{remark}
    Not all unique factorization domains are principal ideal domains.
    (eg. $\mathbb{Z}[x]$ with $2\mathbb{Z}[x]+x\mathbb{Z}[x]$).
\end{remark}


\section{Ring Homomorphisms}


\begin{definition}
    A \textbf{ring homomorphism} for rings $R,S$ is a map $\varphi:R\to S$
    satisfying $\varphi(a+b)=\varphi(a)+\varphi(b)$ and
    $\varphi(ab)=\varphi(a)\varphi(b)$ for all $a,b\in R$.
\end{definition}

\begin{definition}
    If there exists a $\varphi^{-1}:S\to R$ for a ring homomorphism
    $\varphi:R\to S$ such that $\varphi^{-1}(\varphi(a))=a$ for all $a\in R$,
    we call $R$ and $S$ \textbf{isomorphic}.
\end{definition}

\begin{remark}
    We can show the inverse of a bijective ring homomorphism $\varphi:R\to S$
    is also a ring homomorphism, therefore, $R$ and $S$ are isomorphic.
\end{remark}

\begin{remark}
    For ring homomorphism $\varphi:R\to S$, we have $\varphi(0_R)=0_S$,
    $\varphi(-a)=-\varphi(a)$ for all elements $a\in R$, and
    $\varphi(a^{-1})=\varphi(a)^{-1}$ for all units $a\in R$.
    We can also show $\varphi(R)=\{\varphi(r):r\in R\}$ is a subring of $S$.
\end{remark}

\begin{definition}
    The \textbf{kernel} of $\varphi:R\to S$ is defined as
    $\ker(\varphi)=\{\varphi(r)=0_S:r\in R\}$.
\end{definition}

\begin{remark}
    We can show that $\ker(\varphi)$ is an ideal.
\end{remark}

\begin{theorem}
    A ring homomorphism $\varphi:R\to S$ is injective if and only if
    $\ker(\varphi)=\{0_R\}$.
\end{theorem}

\begin{proof}
    Suppose, towards a contradiction, that $\varphi$ is not injective, that is,
    there exists $a,b\in R$ satisfying $\varphi(a)=\varphi(b)$ and $a\neq b$.
    Then, $a-b\in\ker(\varphi)$ but $a-b\neq 0_R$.
\end{proof}

\begin{theorem}
    Let $R$ be a commutative ring with unity and $\varphi:R\to S$ be a
    surjective ring homomorphism. Then, $S$ is an integral domain if and only
    if $\ker(\varphi)$ is a prime ideal.
\end{theorem}

\begin{proof}
    Let $a,b\in S$ with $\varphi(x)=a$ and $\varphi(y)=b$.
    Suppose $\ker(\varphi)$ is a prime ideal.
    Then, $ab=0\implies\phi(x)\phi(y)=0\implies\phi(xy)=0$ so either
    $x\in\ker(\varphi)$ or $y\in\ker(\varphi)$.
    Without loss of generality, let $x\in\ker(\varphi)$ so
    $\varphi(x)=a=0$.

    Suppose $\ker(\varphi)$ is not a prime ideal.
    There exists $xy\in\ker(\varphi)$ such that $x\notin\ker(\varphi)$ and
    $y\notin\ker(\varphi)$.
    Then, $0=\varphi(xy)=\varphi(x)\varphi(y)=ab=0$ but $\varphi(x)\neq 0$ and
    $\varphi(x)\neq 0$.
\end{proof}

\begin{theorem}
    Let $R$ be a commutative ring with unity, $S$ a non-zero ring, and $\varphi:R\to S$ a
    surjective ring homomorphism.
    Then, $S$ is a field if and only if $\ker(\varphi)$ is a maximal ideal.
\end{theorem}

\begin{proof}
    Suppose $\ker(\varphi)$ is maximal.
    Let $b\in S$ be non-zero and define $J=\varphi^{-1}(bS)=\{\varphi(r)\in bS:r\in R\}$.
    We can show that $J$ is an ideal satisfying $\ker(\varphi)\subseteq J\subseteq R$.
    Then, by maximality of $J$, $\ker(\varphi)=J$ or $J=R$.
    However, since $b\neq 0$, $J=R$ so $b$ is a unit.
    It then follows that $S$ is a field.

    Now suppose $\ker(\varphi)$ is not maximal, that is, there exists an ideal $I$ such that
    $\ker(\varphi)\subset I\subset R$.
    Pick $y\in I\setminus\ker(\varphi)$ and assume, for a contradiction, that $S$ is a field.
    Then, there exists $x\in R$ such that $\varphi(x)\varphi(y)=1_S$ so $\varphi(xy)=\varphi(1_R)$.
    It follows that $xy-1_R=r\in\ker(\varphi)$ so $1_R=xy-r\in I$ since $xy,r\in I$.
    Therefore, $I=R$, a contradiction.
\end{proof}

\begin{theorem}[First Isomorphism Theorem]
    \label{thm:isomorphism}
    If $\varphi:R\to S_1$ and $\psi:R\to S_2$ are surjective ring homomorphisms with
    $\ker(\varphi)=\ker(\psi)$, there is a ring isomorphism $\sigma:S_1\to S_2$ where
    $\sigma(\varphi(x))=\psi(x)$ for all $x\in R$.
\end{theorem}

\begin{proof}
    Since $\sigma$ is a composition of ring homomorphisms, it itself is a ring homomorphism.
    Therefore, it suffices to show $\sigma$ is both injective and surjective.

    Let $x,y\in R$ such that $\sigma(\varphi(x))=\sigma(\varphi(y))$.
    Then, $\psi(x)=\psi(y)$ so $\psi(x-y)=0_{S_2}$ and $x-y\in\ker(\psi)$.
    Since $\ker(\psi)=\ker(\varphi)$, $x-y\in\ker(\varphi)$ so $\varphi(x)=\varphi(y)$.
    Therefore, $\sigma$ is injective.

    For any $b\in S_2$, we have $\psi(x)=b$ for some $x\in R$.
    Then, $\sigma(\varphi(x))=b$ so $\sigma$ is surjective.
\end{proof}

\begin{remark}
    It follows from Theorem \ref{thm:isomorphism} that, for a surjective homomorphism
    $\varphi:R\to S$, $S$ is entirely determined (up to isomorphism) by $R$ and $\ker(\varphi)$.
\end{remark}

\begin{definition}
    For ring $R$, ideal $I\subseteq R$, and $a\in R$, the \textbf{coset} of $a$ modulo $I$ is
    defined $a+I=\{a+x:x\in I\}=[a]_I$.
\end{definition}

\begin{theorem}
    \label{thm:cosets disjoint}
    Let $a+I$ and $b+I$ be cosets.
    Then, either $a+I=b+I$ or $(a+I)\cap(b+I)=\emptyset$.
\end{theorem}

\begin{proof}
    Suppose $(a+I)\cap(b+I)\neq\emptyset$ and let $c\in(a+I)\cap(b+I)$.
    Then, $c=a+x=b+y\implies a-b=y-x\in I$.
    Now, for all $a+z\in a+I$, $a+z-(a-b)=z+b\in b+I$.
\end{proof}

\begin{definition}
    The \textbf{quotient ring} for ring $R$ by an ideal $I\subseteq R$ is denoted
    $R/I=\{a+I:a\in R\}$ and represents the collection of cosets modulo $I$.
\end{definition}

\begin{remark}
    It follows from Theorem \ref{thm:cosets disjoint} that the map for $\varphi:R\to R/I$ where
    $a\mapsto a+I$ is well defined.
    Furthermore, we can show $\varphi$ is a surjective ring homomorphism with $\ker(\varphi)=I$.
\end{remark}

\begin{theorem}[Chinese Remainder Theorem]
    Let $R$ be a commutative ring with unity and $I,J\subset R$ be ideals satisfying $I+J=R$.
    Then, $R/(I+J)\cong R/I\times R/J$.
\end{theorem}

\begin{proof}
    We explain the isomorphism via Theorem \ref{thm:isomorphism}.
    Define $\varphi:R\to R/(I\cap J)$ as $\varphi(a)\mapsto a+(I\cap J)$ and
    $\psi:R\to R/I\times R/J$ as $\psi(a)\mapsto(a+I,a+J)$.
    Then, $\ker(\varphi)=\ker(\psi)$ since $c\in(I\cup J)\iff c\in I$ and $c\in J$.
    Also, $\varphi$ is surjective by definition.
    Therefore, it suffices to show $\psi$ is surjective.

    We want that for all $a+I$ and $b+J$, there exists $c\in R$ such that $c+I=a+I$ and $c+J=b+J$.
    Since $I+J=R$, $1_R\in I+J$ so there exists $x\in I$ and $y\in J$ such that $x+y=1$.
    We will choose $c=ax+by$ and show that $c-a\in I$.
    Then, by symmetry, $c-b\in J$ and we are done.
    \[c-a=(ay+bx)-a(x+y)=(b-a)x\in I\]
\end{proof}

\section{Fields and Stuff}

\begin{theorem}
    \label{thm:characteristic ring homomorphism}
    Let $R$ be a ring with unity and define the ring homomorphism $\varphi:\mathbb{Z}\to R$ as
    $\varphi(n)=\underbrace{1_R+\cdots+1_R}_{n\text{ times}}$.
    Then, $\ker(\varphi)=c\mathbb{Z}$ for $c$ the characteristic of $R$.
\end{theorem}

\begin{remark}
    We see, from Theorem \ref{thm:characteristic ring homomorphism}, that every ring with unity and
    characteristic $c$ contains a subring isomorphic to $\mathbb{Z}/c\mathbb{Z}$.
    Furthermore, for field $F$ with characteristic $p$ ($p$ is prime by Theorem
    \ref{thm:composite characteristic implies zero divisor}), $F$ contains a subring of
    $\mathbb{Z}/p\mathbb{Z}$ or $\mathbb{Q}$.
\end{remark}

\begin{definition}
    The \textbf{prime fields} $F_p$ are defined to be $\mathbb{Q}$ and $\mathbb{Z}/p\mathbb{Z}$
    (for any prime $p$).
\end{definition}

\begin{remark}
    Every field contains a prime field.
\end{remark}

\begin{definition}
    A \textbf{field extension} of field $F$ is a field $E$ such that $F\subseteq E$.
\end{definition}

\begin{definition}
    An \textbf{$F$-vector space} for a field $F$ is a commutative group $V$ with scalar
    multiplication between $F$ and $V$ such that, for $a,b\in F$ and $u,v\in V$, $a(u+v)=au+av$,
    $(a+b)v=av+bv$, $(ab)v=a(bv)$, and $\exists 1\in F$ such that $1v=v$.
\end{definition}

\begin{definition}
    For an $F$-vector space $V$, an element $w\in V$ is an \textbf{$F$-linear combination} of
    $v_1,\dots,v_n\in V$ if there exist $a_1,\dots,a_n\in F$ such that
    $a_1v_1+\cdots+a_nv_n=w$.
\end{definition}

\begin{definition}
    We say $U\subseteq V$ \textbf{generates} $V$ over $F$ if every $w\in V$ can be written as an
    $F$-linear combination of $U$.
\end{definition}

\begin{definition}
    For an $F$-vector space $V$, we say that the set $U=\{v_1,\dots,v_n\}\subset V$ is
    \textbf{linearly independent} if, for $a_1,\dots,a_n\in F$, $a_1v_1+\cdots+a_nv_n=0$ implies
    $a_1=\cdots=a_n=0$.
\end{definition}

\begin{definition}
    We say $U\subseteq V$ is an \textbf{$F$-basis} for $V$ if $U$ is linearly independent and
    generates $V$.
\end{definition}

\begin{theorem}
    All finite $F$-bases of $V$ are the same size.
\end{theorem}

\begin{definition}
    If $U=\{v_1,\dots,v_n\}$ is a finite $F$-basis of $V$, we say that $V$ is an
    \textbf{$n$-dimensional} $F$-vector space.
\end{definition}

\begin{remark}
    If $V$ can be generated by a finite basis, it is finite dimensional as an $F$-vector space.
\end{remark}

\begin{theorem}
    A finite field $E$ has a positive prime characteristic $p$ and contains $p^n$ elements for some
    $n$.
\end{theorem}

\begin{theorem}
    Since $E$ is finite, it must contain some prime field $F=\mathbb{Z}/p\mathbb{Z}$, so $E$ is an
    $F$-vector space.
    Since $E$ is finite and generates itself, it must contain a finite basis $\{v_1,\dots,v_n\}$.

    We define $\varphi:F^n\to E$ as $\varphi(a_1,\dots,a_n)=a_1v_1+\dots+a_nv_n$.
    By definition, $\varphi$ is surjective.
    Furthermore, $a_1v_1+\cdots+a_nv_n=b_1v_1+\cdots+b_nv_n$ implies 
    $(a_1-b_1)v_1+\cdots+(a_n-b_n)v_n=0$, so $a_1=b_1,\dots,a_n=b_n$ by the linear independence of
    the basis.
    Therefore, $\varphi$ is also injective.
    Since there are $p^n$ elements in $F^n$, we conclude that there are also $p^n$ elements in $E$
    by the bijectivity of $\varphi$.
\end{theorem}

\begin{definition}
    For $E$ a field extension of $F$, $\alpha\in E$ is \textbf{algebraic} if it is a root of some
    polynomial $g\in F[x]$.
    Otherwise, it is \textbf{transcendental}.
\end{definition}

\begin{definition}
    For $\alpha\in E$ algebraic over $F\subseteq E$, the \textbf{minimal polynomial} of
    $\alpha$ over $F$ is the monic polynomial $P\in F[x]$ of least degree for which $P(\alpha)=0$.
\end{definition}

\begin{theorem}
    \label{thm:minimal polynomial is unique and irreducible}
    The minimal polynomial of $\alpha$ over $F$ is unique and irreducible.
    Furthermore, any $g\in F[x]$ satisfying $g(\alpha)=0$ is divisible by the minimal polynomial.
\end{theorem}

\begin{proof}
    Let $P\in F[x]$ be the minimal polynomial of $\alpha$ over $F$.
    Suppose the $P$ is not unique, so we have another minimal polynomial $Q\in F[x]$.
    Then, $(P-Q)(\alpha)=0$ but $\deg(P-Q)<\deg(P)$, contradicting the minimality of $P$.
    Now suppose $P$ is reducible, then $0=P(\alpha)=f(\alpha)g(\alpha)$.
    However, $x-\alpha$ divides one of $f(x),g(x)$ which contradicts the minimality of $P$.

    Suppose $g\in F[x]$ such that $g(\alpha)=0$.
    Then, by Euclidean Division, $g(x)=q(x)P(x)+r(x)$ with $\deg(r)<\deg(P)$.
    However, $0=g(\alpha)=\cancel{q(\alpha)P(\alpha)}+r(\alpha)=r(\alpha)=0$.
    Therefore, since $\deg(r)<\deg(P)$, $r(x)=0$ by the minimality of $P$.
\end{proof}

\begin{theorem}
    For a field extension $E$ of $F$ and algebraic $\alpha\in E$ over $F$ with minimal polynomial
    $P=a_0x^0+\cdots+a_dx^d\in F[x]$,
    $F(\alpha)=\{u_0\alpha^0+\cdots+u_{d-1}\alpha^{d-1}:u_0,\dots,u_{d-1}\in F\}$ is a
    $d$-dimensional subfield of $E$
    as an $F$-vector space.
\end{theorem}

\begin{proof}
    It is clear that $F(\alpha)$ contains $0$ and is closed under addition and taking additive
    inverses.
    It suffices to show that it is closed under multiplication and taking multiplicative inverses.

    See that $P(\alpha)=0$ so $\alpha^d=-(a_0\alpha^0+\cdots+a_{d-1}\alpha^{d-1})$.
    Then, any expression with a term including $\alpha^m$ where $m>d-1$ can be rewritten as an
    expression on $\alpha^0,\dots,\alpha^{d-1}$, therefore, $F(\alpha)$ is closed over
    multiplication.

    For some $w\in F(\alpha)$, choose $f(x)\in F[x]$ such that $f(\alpha)=w$.
    Then, since $F[x]$ are polynomials over a field, we can use the Euclidean Algorithm on $f,P$ to
    find $g(x),h(x)\in F[x]$ such that $f(x)g(x)+h(x)P(x)=1$.
    Then, $f(\alpha)g(\alpha)+\cancel{h(\alpha)P(\alpha)}=wg(\alpha)=1$ with
    $g(\alpha)\in F(\alpha)$.
\end{proof}

\begin{theorem}
    For an algebraic $\alpha\in E$ over $F$, with minimal polynomial of degree $d$, define
    $\varphi:F[x]\to F(\alpha)$ such that $\varphi(f(x))=f(\alpha)$.
    Then, $\varphi$ is a surjective ring homomorphism with $\ker(\varphi)=\langle P\rangle$ so
    $F(\alpha)$ is isomorphic to $F(\alpha)/\langle P\rangle$.
\end{theorem}

\begin{proof}
    We know that polynomial evaluation is a ring homomorphism.
    We also know $\varphi(F[x])=F(\alpha)$ since $\varphi(F[x])\supseteq F(\alpha)$ trivially and
    every expression containing $\alpha^d$ can be written as an expression on
    $\alpha^0,\dots,\alpha^{d-1}$ so $\varphi(F[x])=F(\alpha)$ which means $\varphi$ is surjective.
    Finally, $\ker(\varphi)=\langle P\rangle$ because every $g\in F[x]$ with $g(\alpha)=0$ is
    divisible by $P$ (Theorem \ref{thm:minimal polynomial is unique and irreducible}).
\end{proof}


\end{document}
