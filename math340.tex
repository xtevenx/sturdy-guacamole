\documentclass{article}

\usepackage{amsmath}
\usepackage{amssymb}
\usepackage{amsthm}
\usepackage{geometry}
\usepackage{parskip}
\usepackage{setspace}
\usepackage{xcolor}

\usepackage{hyperref}

% \geometry{paperheight=113mm, paperwidth=62mm}  % actual size
\geometry{paperheight=212mm, paperwidth=116mm, margin=3mm}

\setstretch{1.26}

\pagecolor{black}
\color{white}

\newtheorem{theorem}{Theorem}[definition]
\newtheorem{corollary}{Corollary}[definition]
\newtheorem{proposition}{Proposition}[definition]

\theoremstyle{definition}
\newtheorem{definition}{Definition}[section]
\newtheorem{subdefinition}{Definition}[definition]

\theoremstyle{remark}
\newtheorem{remark}{Remark}[definition]

\begin{document}


\section{Beginning}
\pagenumbering{gobble}


\begin{definition}
    A \textbf{group} is a pairing of a set and a binary operation such that the
    operation is associative, and each element of the set has both an inverse
    and an identity.
\end{definition}

\begin{remark}
    Considering an identity element is defined both as a left and a right
    identity, it can be proven that it is unique for the group.
\end{remark}

\begin{remark}
    Considering the binary operation is well defined, it can be proven that the
    inverse is distinct for every element of the set.
\end{remark}

\begin{subdefinition}
    A \textbf{commutative group} is a group where the binary operation is
    commutative.
\end{subdefinition}


\begin{definition}
    A \textbf{ring} is a commutative group with another operation defined such
    that the two operations are similar to ``addition'' and ``multiplication''
    of the integers.
    The multiplication operation must be associative and distributive.
\end{definition}

\begin{subdefinition}
    A \textbf{commutative ring} is a ring where multiplication is commutative.
\end{subdefinition}

\begin{subdefinition}
    We say that a ring has \textbf{unity} if there is a multiplicative identity.
\end{subdefinition}

\begin{remark}
    \label{rem:zero multiply}
    For any ring $R$ with additive identity 0, it can be proven that $0a=0$ for
    all $a\in R$.
\end{remark}

\begin{remark}
    Using Remark \ref{rem:zero multiply}, it can be proven that
    $-ab=(-a)b=a(-b)$ for all $a,b\in R$.
\end{remark}

\begin{subdefinition}
    For some non-zero $a,b\in R$, we say $a$ and $b$ are \textbf{zero divisors}
    if $ab=0$.
\end{subdefinition}

\begin{remark}
    \label{rem:zero divisor}
    For some non-zero $a\in R$, it can be proven that $a$ is a left zero
    divisor if and only if there exists non-zero $b,c\in R$ such that $b\neq c$
    and $ab=ac$.
\end{remark}

\begin{remark}
    It follows from Remark \ref{rem:zero divisor}, that if a ring $R$ does
    \textit{not} have any zero divisors, then $ab=ac\implies b=c$ for all
    $a,b,c\in R$ and $a\neq 0$.
\end{remark}

\begin{subdefinition}
    A \textbf{unit} in a ring with unity is an element which has a
    multiplicative inverse.
\end{subdefinition}

\begin{remark}
    A unit cannot be a zero-divisor.
\end{remark}


\begin{definition}
    An \textbf{integral domain} is a commutative ring with unity and no zero
    divisors.
\end{definition}


\begin{definition}
    A \textbf{field} is a commutative ring where every non-zero element is a
    unit, and the additive and multiplicative identities are not equal.
\end{definition}


\section{Developing}


\begin{definition}
    The \textbf{characteristic} of a ring is the lowest integer $c$ such that
    $\underbrace{1+1+\cdots+1}_{c\text{ times}}=0$.
\end{definition}

\begin{theorem}
    If the characteristic of a ring is composite, it must have zero divisors.
\end{theorem}

\begin{proof}
    Let $c$ be the characteristic of some ring where there exists positive
    integers $m,n$ such that $c=mn$ and $m,n<c$.
    Consider, using the distributivity of multiplication, that
    $(\underbrace{1+1+\cdots+1}_{m\text{ times}})(\underbrace{1+1+\cdots+1}_{n\text{ times}})=0$.
\end{proof}

\begin{theorem}[Euler's Theorem]
    Let $R^*$ be the finite set of the units in a ring.
    For all $a\in R^*$, $a^{|R^*|}=1$.
\end{theorem}

\begin{proof}
    We have $R^*=\{r_1,\dots,r_n\}=\{ar_1,\dots,ar_n\}$ since multiplication
    is one-to-one.
    Then, $r_1\cdots r_n=(ar_1)\cdots (ar_n)=a^n(r_1\cdots r_n)\implies a^n=1$.
\end{proof}

\begin{theorem}
    \label{thm:finite ring with unity}
    For a finite ring with unity, any element is either 0, a zero divisor, or a
    unit.
\end{theorem}

\begin{proof}
    For an element $r$ that is not zero or a zero divisor, we have the
    following set of non-zero elements $\{r,r^2,\dots\}$.
    Since the ring is finite, we have $r^{e_1}=r^{e_2}$ for some $e_1<e_2$.
    Then, $r^{e_1}=r^{e_2}=r^{e_1}r^{e_2-e_1}\implies r^{e_2-e_1}=1$.
    Therefore, $r\cdot r^{e_2-e_1-1}=1$.
\end{proof}

\begin{remark}
    Theorem \ref{thm:finite ring with unity} shows every finite integral domain
    is a field.
\end{remark}


\begin{definition}
    A \textbf{quadratic ring} extension $R[\gamma]$ of some ring $R$ is created
    by adding an element $\gamma$ to $R$ such that $\gamma^2=c$ for some
    $c\in R$ and $\gamma\notin R$.
\end{definition}

\begin{remark}
    Elements in $R[\gamma]$ are denoted $a+\gamma b$ for $a,b\in R$.
    This means elements in $R[\gamma]$ can be seen as elements in $R\times R$.
\end{remark}

\begin{theorem}
    \label{thm:norm map}
    The norm map\footnotemark $N:R[\gamma]\to R$ is defined as
    $N(a+\gamma b)=a^2-cb^2$ and has the property that $N(a+\gamma b)$ is a
    unit in $R$ if and only if $a+\gamma b$ is a unit in $R[\gamma]$.
    \footnotetext{We have not yet formally defined a \textit{norm map}.}
\end{theorem}

\begin{proof}
    We see that $N(a+\gamma b)^{-1}$ exists if and only if $N(a+\gamma b)$ is a unit.
    Then, $(a+\gamma b)(a-\gamma b)=N(a+\gamma b)$ so
    $(a+\gamma b)\left[(a-\gamma b)N(a+\gamma b)^{-1}\right]=1$.
\end{proof}

\begin{remark}
    Theorem \ref{thm:norm map} shows the quadratic ring extension of any field
    or integral domain maintains that status.
\end{remark}

\begin{definition}
    An element in an integral domain is called \textbf{irreducible} if it
    cannot be written as a product of two non-units.
\end{definition}

\begin{definition}
    Elements $a,b$ in an integral domain $R$ are called \textbf{associates} if
    there exists $u\in R$ such that $a=ub$.
\end{definition}

\begin{definition}
    \label{def:unique factorization}
    An integral domain has \textbf{unique factorization} if every element can
    be written as a product of irreducibles which are unique up to order and
    associates.
\end{definition}

\begin{theorem}
    \label{thm:unique factorization}
    A ring $R$ has unique factorization if all irreducible elements are prime.
\end{theorem}

\begin{proof}
    Let $x=a_1\cdots a_n=b_1\cdots b_m$.
    Since $a_1$ is prime, we know that it divides one of $b_i$.
    Without loss of generality, let $b_1=ca_1$.
    However, since $b_1$ is irreducible and $a_1\neq 1$, we have $c=1$.
    Then, we can repeat this process on $a_2\cdots a_n=b_2\cdots b_m$.
\end{proof}

\begin{definition}
    A \textbf{polynomial ring} $R[x]$ of some ring $R$ is created by using
    polynomials of the variable $x$ using coefficients from $R$.
\end{definition}

\begin{remark}
    For some field $F$, Euclidean division works on $F[x]$ because all non-zero
    coefficients are units.
    It then follows that irreducible elements are prime, so unique factorization
    exists in $F[x]$.
\end{remark}

\begin{theorem}[Fundamental Theorem of Algebra]
    \label{thm:fundamental theorem of algebra}
    The only irreducible polynomials in $\mathbb{C}[x]$ are linear.
\end{theorem}

\begin{remark}
    It follows from Theorem \ref{thm:fundamental theorem of algebra} that the
    only irreducible polynomials in $\mathbb{R}[x]$ are linear or quadratic.
    This can be proven using $\mathbb{R}[x]\subset\mathbb{C}[x]$ and that
    multiplying some linear $f(x)\in\mathbb{C}[x]$ with its conjugate results
    in some $F(x)\in\mathbb{R}[x]$ with $\deg(F(x))=2$.
\end{remark}

\begin{definition}
    A subset $I$ of ring $R$ is called an \textbf{ideal} if for all
    $a,b\in I$ and $r\in R$, $a+b,-a,ra,ar\in I$.
\end{definition}

\begin{definition}
    For a commutative ring $R$ and $a\in R$, a \textbf{principal ideal}
    generated by $a$ is defined as $aR=\{ar:r\in R\}$.
    For some $a,b\in R$, we can also generate $(a,b)R=\{xa+yb:x,y\in R\}$.
\end{definition}

\begin{remark}
    For $a\in R$ with integral domain $R$, $aR=1R=R$ if and only if $a$ is a
    unit.
\end{remark}

\begin{remark}
    For $a,b\in R$, $b\mid a\implies bR\subseteq aR$.
    Furthermore, $aR=bR$ if and only if $a$ and $b$ are associates.
\end{remark}

\begin{remark}
    For $a,b\in R$, if $a$ is irreducible and $b\mid a$, then
    $aR\subseteq bR\subseteq R$ so either $aR=bR$ or $bR=R$.
    Therefore, $aR$ is not properly contained in any other principal ideal.
    Also, if $a$ is not irreducible, then $aR\subset bR\subset R$.
\end{remark}

\begin{theorem}
    If an element $a\in R$ cannot be written as a finite product of
    irreducibles, then $R$ has an infinite ascending chain of principal ideals.
\end{theorem}

\begin{proof}
    Assume that $a$ cannot be written as a finite product of irreducibles.
    Then, $a=r_1a_1=r_1r_2a_2=\dots$ for non-units $r_i,a_i$ and reducible $a_i$.
    This implies $aR\subset a_1R\subset a_2R\subset\cdots$.
\end{proof}

\begin{remark}
    This tells us that every element in $\mathbb{N}$ has a factorization into
    irreducibles since every proper divisor is ``smaller'' so there cannot be
    an infinite chain.
\end{remark}

\begin{definition}
    An integral domain $R$ is a \textbf{principal ideal domain} if every ideal
    in $R$ is a principal ideal.
\end{definition}

\begin{proposition}
    \label{thm:integers is principal ideal domain}
    The ring $\mathbb{Z}$ is a principal ideal domain.
\end{proposition}

\begin{proof}
    If $I=\{0\}$, then $I=0\mathbb{Z}$.
    Therefore, we prove with $I\neq\{0\}$.
    Then, there exists a positive element in $I$.
    Let $a$ be the the least positive element in $I$ and we claim that
    $I=a\mathbb{Z}$.

    Let $b\in I$ be some other element in $I$.
    Then we have $b = qa + r$ for $0\leq r<a$.
    This also means $b-qa=r$ so $r\in I$.
    However, by the minimality of $a$, this implies $r=0$ so $b$ is a multiple
    of $a$ and $b\in a\mathbb{Z}$.
\end{proof}

\begin{corollary}
    For field $F$, $F[x]$ is a principal ideal domain.
\end{corollary}

\begin{theorem}
    For a principal ideal domain, every ascending chain of ideals stabilizes.
\end{theorem}

\begin{proof}
    Let $I_1\subseteq I_2\subseteq\cdots$ be an ascending chain of ideals in a
    principal ideal domain $R$.
    Then, $\bigcup_{i=1}^\infty I_i$ is a principal ideal $aR$.
    For some $j$, $a\in I_j$ so $aR=I_j=I_{j+1}=\cdots$.
\end{proof}

\begin{definition}
    Let $I,J$ be ideals of $R$. Then, $I+J$ is the smallest ideal which contains
    both $I$ and $J$.
    Therefore, $I+J=\{a+b:a\in I\text{ and }b\in J\}$.
\end{definition}

\begin{remark}
    Since $\mathbb{Z}$ is a principal ideal domain, $a\mathbb{Z}+b\mathbb{Z}=d\mathbb{Z}$.
    Then, $d\mathbb{Z}=\{xa+yb:x,y\in\mathbb{Z}\}$.
    Therefore, $d=\gcd(a,b)$ since it is the least positive element (by proof
    of Theorem \ref{thm:integers is principal ideal domain}).
\end{remark}

\begin{definition}
    An ideal $I$ of ring $R$ is a \textbf{prime ideal} if $ab\in I$ implies
    $a\in I$ or $b\in I$ for all $a,b\in R$.
\end{definition}

\begin{remark}
    An element $p\in R$ is prime if and only if $pR$ is prime.
\end{remark}

\begin{remark}
    Not all prime ideals are principal (eg. $(x,y)\subset\mathbb{Q}[x,y]$).
\end{remark}

\begin{definition}
    An ideal $I$ in ring $R$ is called \textbf{maximal} if for any ideal
    $J\subseteq R$ where $I\subseteq J\subseteq R$, it follows that $I=J$ or
    $J=R$.
\end{definition}

\begin{remark}
    In a principal ideal domain, the principal ideal generated by an
    irreducible element is maximal.
\end{remark}

\begin{theorem}
    In an integral domain, maximal ideals are prime.
\end{theorem}

\begin{proof}
    Let $I$ be a maximal ideal of ring $R$ with $bc\in I$ and $b\notin I$.
    Then, we have $I\subsetneq I+bR\subseteq R$ so, by the maximality of $I$,
    $I+bR=R$.
    This also means that $1\in I+bR$ so $1=a+br$ for $a\in I$ and $r\in R$.
    Multiplying through by $c$, this gives us $c=ac+bcr\in I$ since $a,bc\in I$.
\end{proof}

\begin{remark}
    For a principal ideal domain $R$, this gives us that $a\in R$ is
    irreducible implies $aR$ is maximal implies $aR$ is prime implies a is prime.
    Therefore, by Theorem \ref{thm:unique factorization}, every principal ideal
    domain has unique factorization.
\end{remark}

\begin{definition}
    A ring is a unique factorization domain if every non-zero non-unit can be
    written uniquely as a product of irreducible elements, up to order and
    associates.
    Duplicate of Definition \ref{def:unique factorization}; don't ask why.
\end{definition}

\begin{remark}
    Unique factorization domains exist which are not principal ideal domains.
    For example, $\mathbb{Z}[x]$ with $2\mathbb{Z}[x]+x\mathbb{Z}[x]$.
\end{remark}


\end{document}
